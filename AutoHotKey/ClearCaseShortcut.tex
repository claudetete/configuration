\documentclass[ignorenonframetext]{beamer}

\usepackage[francais]{babel}
\usepackage{ucs}
\usepackage[utf8x]{inputenc}
\usepackage[T1]{fontenc}
\usepackage{hyperref}

% to set picture width
\def\myincludegraphics{
  % pictures take 70% of text area
  \includegraphics[width=0.5\textwidth]
}

% title page block rounded and shadow
\useinnertheme[shadow]{rounded}
% add footer (need to be before theme)
\useoutertheme{infolines}
% inherit theme
\usetheme[height=7mm]{Rochester}
% emacs icon color
\usecolortheme[RGB={110,108,183}]{structure}
% use ball instead of rectangle
\setbeamertemplate{items}[ball]
% remove navigation line
\setbeamertemplate{navigation symbols}{}

% get title, author and date
\title{Présentation de ClearCaseShortcut}
\author{Claude TÊTE}
\date{\today}


% start document
\begin{document}

% add the title page
\frame{\titlepage}
\begin{frame}{Sommaire}
  \tableofcontents
\end{frame}

\section{Introduction}
% new page
\begin{frame}[fragile]
  \frametitle{Introduction}
  \begin{itemize}
  \item ClearCaseShortcut est un script AutoHotKey.
  \item Des raccourcis homogènes dans chaque fenêtre ou application pour des fonction de ClearCase\copyright IBM.
  \item Cela fonctionne avec MS Explorer, ClearCase History Browser, ClearCase Find checkout, ClearCase Explorer, ClearCase Tree Version, UltraEdit and Notepad++.
  \end{itemize}
\end{frame}

\section{Présentation}
% new page
\begin{frame}[fragile]
  \frametitle{Présentation 1/3}
  \begin{itemize}
  \item Le script vient se loger dans la barre des notifications avec l'icône de ClearCase Explorer.
  \item Un clic droit dessus pour faire apparaître un menu pour quitter le script, éditer le fichier de configuration ou ouvrir les options.
    \begin{figure}[h]
      \centering\myincludegraphics{./img/Menu.png}
      \caption{Menu}
    \end{figure}
  \end{itemize}
\end{frame}

% new page
\begin{frame}[fragile]
  \frametitle{Présentation 2/3}
  \begin{itemize}
  \item Raccourcis claviers, par exemple pour ouvrir le \og Version Tree\fg{} d'un fichier depuis MS Explorer :
    \begin{itemize}
    \item Selectionner les éléments voulus.
    \item Touche Windows + C puis T (sans la touche Windows)
    \item Chaque \og Version Tree Browser\fg{} correspondant à chaque fichier selectionné va s'ouvrir.
    \end{itemize}
  \item Pour la liste complète des raccourcis ouvrir \og Help\fg{} ou le fichier README.
  \end{itemize}
\end{frame}

% new page
\begin{frame}[fragile]
  \frametitle{Présentation 3/3}
  \begin{itemize}
  \item Un \textbf{long} clic droit sur un élément va ouvrir un menu contextuel sans accéder au réseau (donc instantané):
    \begin{figure}[h]
      \centering\myincludegraphics{./img/ContextMenu.png}
      \caption{Menu Contextuel}
    \end{figure}
  \end{itemize}
\end{frame}


\end{document}

%%% Local Variables:
%%% mode: latex
%%% TeX-master: t
%%% End:
